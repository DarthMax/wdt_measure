\documentclass[fontsize=12pt, paper=a4, headinclude, twoside=false, parskip=half+, pagesize=auto, numbers=noenddot, open=right, toc=listof, toc=bibliography]{scrreprt}
\usepackage[left=3cm, bottom=3cm, top=3cm]{geometry} % wenn es nicht anders geht sonst typearea (unten)
 \usepackage{multirow} % Tabellen-Zellen über mehrere Zeilen
\usepackage{multicol} % mehre Spalten auf eine Seite
\usepackage{tabularx} % Für Tabellen mit vorgegeben Größen
\usepackage[automark]{scrpage2} % Kopf- und Fußzeilen
\usepackage[T1]{fontenc} % Ligaturen, richtige Umlaute im PDF 
 \usepackage[utf8]{inputenc}% UTF8-Kodierung für Umlaute usw
 
 
\begin{document}
%\pagestyle{empty}

%++++++++++++++++++++++++++++++++++++++++++++++++++++++++++++++++++
% Titelseite
%\clearscrheadings\clearscrplain
\begin{center}
{\large Universität Leipzig}\\
{\large Fakultät für Mathematik und Informatik}\\
{\large Institut für Informatik}\\
{\large Abteilung Automatische Sprachverarbeitung}\\
\vspace{5mm}
\begin{Large}
\textbf{Wortschatz Zeitgeist}\\
\vspace{5mm}
Seminararbeit\\
\vspace{5mm}
\end{Large}
\vspace{5mm}
\begin{tabular}{ r l }
{\bf Autoren:} 	& TBA\\
			& Kießling, Max\\
			& Otto, Wolfgang (2885214)\\
{\bf Modul:} & Anwendungen Linguistische Informatik (10-202-2307)\\
{\bf Abgabe:} & {\today}. (Sommersemester 2015)\\
{\bf Betreuer:} & Maciej Janicki\\
{\bf Seminarleiter:}&Prof. Dr. Uwe Quasthoff\\ 
\end{tabular}\\
\end{center}
\clearpage

%++++++++++++++++++++++++++++++++++++++++++++++++++++++++++++++++++
% Inhaltsverzeichnis
%\pagestyle{useheadings} % normale Kopf- und Fußzeilen für den Rest
%\pagenumbering{Roman} % große Römische Seitenummerierung
\tableofcontents
