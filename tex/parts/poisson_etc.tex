\subsection{Relatives Vorkommen (Referenz)}
\emph{Idee: } Tokens, deren relatives Auftreten am gewählten Tag im Verhältnis zum relativen Auftreten im Referenzzeitaum (2014) besonders groß ist, sind interessante Wörter.\\ 
	\emph{Formel: } 
	\begin{equation}
	sig_{freqratio}(w) = \frac{\frac{k_{day}}{n_{day}}}{\frac{k_{2014}}{n_{2014}}}
	\end{equation}
	$k_{day}$: Frequenz des Tokens an einem Tag\\
	$n_{day}$: Summe der Frequenzen aller Tokens eines Tages\\
	$k_{2014}$: Frequenz des Tokens im Referenz Zeitrahmen (2014)\\
	$n_{day}$: Summe der Frequenzen aller Tokens im Referenzzeitrahmen (2014)\\
\subsection{Poisson-Maß}

Die Formel leitet sich aus der Poissonverteilung ab und beschreibt wie Wahrscheinlich es ist, dass die gemessene Tagesfrequenz beobachtet werden kann. 
\begin{equation}
sig_{poisson}(w) = \frac{k(\log(k)-\log(n\cdot p) -1 ))}{\log(n)}
\end{equation}
k:= Anzahl der Token von w in Tagesbericht\\
n := Anzahl der Tokens in Tagesbericht\\
p := relativer Anteil eines Tokens am Jahreskorpus\\
Es ist das gleiche Maß wie in \cite[S. 338-340]{heyer06} beschrieben und hergeleitet. Hier aber nicht zum auffinden von signifikanten Kookurenzen, sondern zum auffinden von signifikanten Nennungen im Tageskorpus gegenüber einem Vergleichskorpus.\\

\subsection{Termfrequenz inverse Dokumentenfrequenz (tf-idf)}
 \begin{equation}
sig_{tf idf}(w) = \frac{k}{\max(K)} \cdot \log ( \frac{365}{|documentdays(w)|})
\end{equation}

\subsection{Termfrequenz inverse Dokumentenfrequenz inverse Quellenfrequenz (tf-idf-isf)}
\emph{Idee: } W\"orter sind dann interessant, wenn sie an einem Tag in m;glichst vielen verschiedenen Quellen gennant werden.\\
Als Quelle definieren wir die Serveradresse einer Quelle. Diese wird mittels eines regul\"aren Ausdrucks aus den zugeordneten Quellen in der MySQL-Datenbank ermittelt. Als Gesamtzahl der Quellen verwenden wir alle an einem Tag den W\"ortern zugeordnete Quellen.\\
Das entstandene Signifikanzma\ss wird wie folgt definiert:
 \begin{equation}
sig_{tf idf isf}(w) = sig_{tf idf}(w) \cdot \log ( \frac{Q_d}{q_d(w)})
\end{equation}
Analog zur inversen Dokumentenfrequenz wird also das tf-idf-Signifikanzma\ss  mit dem Logarithmus der inversen relativen Anzahl der Quellenfrequenz multipliziert. $Q_d$ ist die Anzahl aller erw\"ahnten Quellen an einem Tag $d$ und $q_d()$ bildet ein Wort auf die Anzahl der Quellen ab, in denen das Wort an Tag $d$  erw\"ahnt wird. 
