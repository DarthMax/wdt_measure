\documentclass[fontsize=12pt, paper=a4, headinclude, twoside=false, parskip=half+, pagesize=auto, numbers=noenddot, open=right, toc=listof, toc=bibliography]{scrreprt}
\usepackage[left=3cm, bottom=3cm, top=3cm]{geometry} % wenn es nicht anders geht sonst typearea (unten)
\usepackage{multirow} % Tabellen-Zellen über mehrere Zeilen
\usepackage{multicol} % mehre Spalten auf eine Seite
\usepackage{tabularx} % Für Tabellen mit vorgegeben Größen
\usepackage[automark]{scrpage2} % Kopf- und Fußzeilen
\usepackage[T1]{fontenc} % Ligaturen, richtige Umlaute im PDF 
\usepackage[utf8]{inputenc}% UTF8-Kodierung für Umlaute usw
\usepackage{bibgerm} % Bibliographie Umlaute in BibTeX
\usepackage{mathtools}
\usepackage{graphicx}
\usepackage[autostyle=true,german=quotes]{csquotes} %deutsche anfuehrungszeichen
\usepackage{hyperref}
\usepackage{listings}
\usepackage{gensymb}
\usepackage{xcolor}


\lstset{
	breaklines=true, 
	postbreak=\raisebox{0ex}[0ex][0ex]{\ensuremath{\color{red}\hookrightarrow\space}},
  	literate={ö}{{\"o}}1
           {ä}{{\"a}}1
           {ü}{{\"u}}1
           {°}{{\degree}}1
}
 
 \newcommand{\ourtitlepage}{
 %++++++++++++++++++++++++++++++++++++++++++++++++++++++++++++++++++
% Titelseite
%\clearscrheadings\clearscrplain
\begin{center}
{\large Universität Leipzig}\\
{\large Fakultät für Mathematik und Informatik}\\
{\large Institut für Informatik}\\
{\large Abteilung Automatische Sprachverarbeitung}\\
\vspace{5cm}
\begin{Large}
\textbf{Vergleich von Maßen zum Finden von tagesaktuellen Wörtern}\\
\vspace{5mm}
Seminararbeit\\
\vspace{5mm}
\end{Large}
\vspace{9cm}
\begin{tabular}{ r l }
{\bf Autoren:}	& Kießling, Max\\
			& Otto, Wolfgang (2885214)\\
			& (Döring, Thomas)\\
{\bf Modul:} & Anwendungen Linguistische Informatik (10-202-2307)\\
{\bf Abgabe:} & {\today}. (Sommersemester 2015)\\
{\bf Betreuer:} & Maciej Janicki\\
{\bf Seminarleiter:}&Prof. Dr. Uwe Quasthoff\\ 
\end{tabular}\\
\end{center}
\thispagestyle{empty}
\clearpage


}


\begin{document}
\ourtitlepage 
\tableofcontents
\pagenumbering{roman} % Inhaltsverzeichnis roemisch 
\clearpage
\pagenumbering{arabic} % ab jetzt die normale arabische Nummerierung



% EINLEITUNG ###################################################################################
\chapter{Einleitung}

%##########################################
\section{Aufgabenstellung}
Das Portal  \emph{Wörter des Tages}\footnote{\url{http://wortschatz.uni-leipzig.de/wort-des-tages}, Abgerufen am 29.10.2015,~9:34~Uhr} stellt eine Übersicht von Wörtern, die an einem ausgewählten Tag besonders relevant erschienen dar (Siehe Abbildung~\ref{pic.homepage}). Die Wörter sind in neun Kategorien eingeordnet. Nach der Beschreibung auf der Website werden die Wörter ermittelt in dem die tagesaktuelle Häufigkeit eines Wortes mit seiner durchschnittlichen Häufigkeit über längere Zeit hinweg gemessen wird.\\
Die Aufgabe dieser Arbeit ist es verschiedene Möglichkeiten der Bestimmung von Wörtern, die an einem gewählten Tag von besonderer Relevanz sind zu beschreiben, zu vergleichen und zu evaluieren. 
Die Datengrundlage zur Erstellung der Wörter des Tages ist ein Corpus, das durch tägliches crawlen von Newsseiten generiert wird. Die Quellen der Newseiten sind eine definierte Menge an für relevant erachtete Seiten mit Nachrichten wie zum Beispiel \emph{Spiegel.de}.\\
Bei allen Ansätzen, die auf das Vorkommen in einem Referenzzeitraum rekurrieren besteht das Korpus aus allen gecrawlten Newsseiten des voragngegangenen Jahres (2014).
Als Zusatzaufgabe soll ein musterbasiertes Verfahren in SQL entworfen werden, das es ermöglicht aufgrund eines gewählten Verfahrens falsch identifizierte Wörter zu filtern. Ein Beispiel hierfür sind Datumsangaben, die als relevant erscheinen, da sie Tagesaktuell oft auftauchen, aber im Vergleichszeitraum selten.

%##########################################
\section{Vergleichbare Ansätze}
Das Problem der Trenderkennung ist ein vielfältiges Problem mit einer großen Bandbreite an Anwendungsgebieten. Eine der vielleicht populärsten Anwendungen ist die Erkennung von Trendbegriffen bei Twitter, einem Microbloging-Dienst mehreren hundert Millionen Nutzern und 500 Mio Tweets täglich. Aufgrund dieser immensen Vielfalt an Nutzern und Nachrichten ist es möglich, dass hochaktuelle Nachrichten und Neuigkeiten rasend schnell global verbreitet werden können. Mithilfe von Trendanalyse ist es möglich globale aber auch lokale Entwicklungen zeitig zu erfassen und zu analysieren.
Einen ähnlichen Ansatz verfolgt das Google Projekt \emph{Google Trends}. Hier werden die Suchenanfragen der weltweit größten Websuchmaschine ausgewertet, wodurch die zeitabhängige Auswertung einzelner Suchbegriffe möglich wird. \\
Aber auch bei kleinerer Datenmengen ist das Erkennen von Trends bzw. Anomalien nützlich. Angewand auf Logdateien ist es so zum Beispiel möglich Angriffe auf ein Computersystem zu erkennen. \cite{Zwietasch14}

\begin{figure}[h!]
    \centering
    \includegraphics[width=1\textwidth]{pictures/wdt_homepage.png}
    \caption{Hompage des Projekts \emph{Wörter des Tages} des Wortschatz Projekts der Universität Leipzig vom 31.10.2015}\label{pic.homepage}
\end{figure}


% HAUPTTEIL THEORIE ##########################################################################
%\chapter{Methoden zum Finden tagesaktueller Wörter}

%##########################################
\chapter{Maße zur Trend-Detection}
Im folgenden Abschnitt werden fünf Methoden vorgestellt, die f\"uer jedes Wort eines Tageskorpus eine Maßzahl bestimmen, die die Relevanz des Wortes an diesem Tag ausdrücken soll.

%#####################
\subsection{Relatives Vorkommen (Referenz)}
\emph{Idee: } Tokens, deren relatives Auftreten am gewählten Tag im Verhältnis zum relativen Auftreten im Referenzzeitaum (2014) besonders groß ist, sind interessante Wörter.\\ 
	\emph{Formel: } 
	\begin{equation}
	sig_{freqratio}(w) = \frac{\frac{k_{day}}{n_{day}}}{\frac{k_{2014}}{n_{2014}}}
	\end{equation}
	$k_{day}$: Frequenz des Tokens an einem Tag\\
	$n_{day}$: Summe der Frequenzen aller Tokens eines Tages\\
	$k_{2014}$: Frequenz des Tokens im Referenz Zeitrahmen (2014)\\
	$n_{day}$: Summe der Frequenzen aller Tokens im Referenzzeitrahmen (2014)\\
\subsection{Poisson-Maß}

Die Formel leitet sich aus der Poissonverteilung ab und beschreibt wie Wahrscheinlich es ist, dass die gemessene Tagesfrequenz beobachtet werden kann. 
\begin{equation}
sig_{poisson}(w) = \frac{k(\log(k)-\log(n\cdot p) -1 ))}{\log(n)}
\end{equation}
k:= Anzahl der Token von w in Tagesbericht\\
n := Anzahl der Tokens in Tagesbericht\\
p := relativer Anteil eines Tokens am Jahreskorpus\\
Es ist das gleiche Maß wie in \cite[S. 338-340]{heyer06} beschrieben und hergeleitet. Hier aber nicht zum auffinden von signifikanten Kookurenzen, sondern zum auffinden von signifikanten Nennungen im Tageskorpus gegenüber einem Vergleichskorpus.\\

\subsection{Termfrequenz inverse Dokumentenfrequenz (tf-idf)}
 \begin{equation}
sig_{tf idf}(w) = \frac{k}{\max(K)} \cdot \log ( \frac{365}{|documentdays(w)|})
\end{equation}

\subsection{Termfrequenz inverse Dokumentenfrequenz inverse Quellenfrequenz (tf-idf-isf)}
\emph{Idee: } W\"orter sind dann interessant, wenn sie an einem Tag in m;glichst vielen verschiedenen Quellen gennant werden.\\
Als Quelle definieren wir die Serveradresse einer Quelle. Diese wird mittels eines regul\"aren Ausdrucks aus den zugeordneten Quellen in der MySQL-Datenbank ermittelt. Als Gesamtzahl der Quellen verwenden wir alle an einem Tag den W\"ortern zugeordnete Quellen.\\
Das entstandene Signifikanzma\ss wird wie folgt definiert:
 \begin{equation}
sig_{tf idf isf}(w) = sig_{tf idf}(w) \cdot \log ( \frac{Q_d}{q_d(w)})
\end{equation}
Analog zur inversen Dokumentenfrequenz wird also das tf-idf-Signifikanzma\ss  mit dem Logarithmus der inversen relativen Anzahl der Quellenfrequenz multipliziert. $Q_d$ ist die Anzahl aller erw\"ahnten Quellen an einem Tag $d$ und $q_d()$ bildet ein Wort auf die Anzahl der Quellen ab, in denen das Wort an Tag $d$  erw\"ahnt wird. 


%#####################
\subsection{Z-Score}
Das bei diesem, von Benattar et al. \cite{benattar2011trend} beschriebenen, Ansatz zu Grunde liegende statistische Mittel ist der Z-Score. Dieser misst die Abweichung einer Zufallsvariable vom Erwartungswert in Vielfachen der Standartabweichung. Als Zufallsvariable dient die relative Worthäufigkeit am jeweiligen Tag. Erwartungswert und Standartabweichung werden mittels des Referenzkorpus berechnet. 

\begin{itemize}
	\item{Wortfrequenz}
		\begin{align*}
			f(w)_d :&= \text{Anzahl der Vorkommen von Wort $w$ an Datum $d$}
		\end{align*}
		
	\item{relative Worthäufigkeit}\\
		Die relative Worthäufigkeit $p(w)_d$ berechnet sich durch:
		\begin{align*}
			t_d   :&= \text{Anzahl verschiedener Worte an Datum $d$} \\
			p(w)_d &= \frac{f(w)_d}{t_d}
		\end{align*}						
		
	\item{Erwartungswert}\\
		Der Erwartungswert $\bar{w}$ berechnet sich durch:
		\begin{align*}
			N      :&= \text{Anzahl der Tage im Referenzkorpus} \\
			\bar{w} &= \frac{1}{N} \sum p(w)_d
		\end{align*}
		
	\item{Standartabweichung}\\
		Die Standartabweichung $\sigma_w$ berechnet sich durch:
		\begin{align*}
			\sigma_w &= \sqrt{\frac{1}{N} \sum (p(w)_d - \bar{w}^2}
		\end{align*}
		
	\item{Z-Score}\\
		Der Z-Score $Z(w)_d$ misst die Abweichung der relativen Worthäufigkeit vom Erwartungswert in Vielfachen der Standartabweichung.
		\begin{align*}
			Z(w)_d &= \frac{p(w)_d - \bar{w}}{\sigma_w}
		\end{align*}
		
\end{itemize}

Ein Problem bei der Verwendung des Z-Scores sind Worte mit sehr kleinem Erwartungswert. Dies sind häufig Worte, welche an nur sehr wenigen Tagen und im schlimmsten Fall gar nicht im Referenzkorpus auftreten. Bei diesem Worten bedeutet bereits eine sehr geringe Wortfrequenz von ein oder zwei Vorkommen einen enormen Ausschlag im Z-Score. Um dieses sogenannte Zero-Frequency-Problem abzuschwächen schlagen Benattar et. al. vor die Worte anhand ihrer Auftrittshäufigkeit zu clustern. Den Clustern werden dabei Z-Score-Schwellwerte zugeordnet. Überschreitet der Z-Score eines Wortes den Schwellwert seines Clusters wird dieses Wort als signifikant und somit als Trend eingestuft. Cluster mit niedriger Auftrittshäufigkeit erhalten dabei höhere Schwellwerte. Je häufiger ein Wort auftritt desto niedriger wird der Schwellwert. 

\begin{itemize}
	
	\item{Auftrittshäufigkeit}\\
		Die Auftrittshäufigkeit $Po(w)$ gibt an wie vielen Tagen innerhalb des betrachteten Zeitraums das Wort mindestens einmal auftritt:
		\begin{align*}
		 	nbD(w) :&= \text{Anzahl der Tage an denen w vorkommt} \\
			c_d	   :&= \text{Anzahl der Tage innerhalb des betrachteten Zeitraums}\\
			  Po(w) &=\frac{nbD(w)}{c_d}
		\end{align*}

	\item{Schwellwerte}\\
		Die Tabelle zeigt den Z-Score-Schwellwert gecluster nach Auftrittshäufigkeit $Po(w)$:\\
		\begin{tabular}{|c|c|c|c|c|c|c|c|c|}
			\hline
			[0-5[ & [5-10[ & [10-20[ & [20-30[ & [30-50[ & [50-60[ & [60-70[ & [70-80[ & [80-100] \\
			\hline
			20 & 25 & 15 & 12 & 10 & 9 & 8 & 6 & 5 \\
			\hline
		\end{tabular}
			
\end{itemize}

%#####################
%\subsection{Weitere Maße}
%Einbeziehung der Anzahl von Quelle

%##########################################
\chapter{Zeitreihenanalysen}
Es wurde eine Pipeline in R geschrieben um geglättete Tagesfrequenzen zu berechnen.\footnote{Dieser Teil des Projektes wurde von Thomas Döring verfasst, der leider nicht in der Lage war bis Ende Oktober einen Beitrag zu dieser Arbeit zu leisten.} Diese Pipeline umfasst sechs Teile. Die Programmteile sind unter thomas/v2/R/1.r~-~6.r im Projektordner zu finden.
\begin{enumerate}
\item Aufbau einer Verbindung mit der Datenbank
\item Laden der Daten aus der Datenbank
\item Reshape der Daten I: Eine Spalte entspricht einem Wort
\item Berechnung der neuen Werte mittels Filter
\item Reshape der Daten II: Umformung in die Ursprüngliche Form mit einem Wort pro Zeile (melt)
\item Schreiben der berechneten Daten in die Datenbank 
\end{enumerate}
Die Abbildung~\ref{pic.time_airplane} stellt beispielhaft die Ergebnisse für das Wort Flugzeug dar. Mit der genutzten Bibliothek kann leicht durch Parameter die größe des Fensters angepasst werden.\footnote{Verwendete Funktion in R dokumentiert in: \url{https://stat.ethz.ch/R-manual/R-devel/library/stats/html/filter.html}, Abgerufen am 30.10.2015, 18:55 Uhr.}
% Hier Beispiel Flugzeug
\begin{figure}[h!]
    \centering
    \includegraphics[width=0.82\textwidth]{pictures/timeFlugzeug.png}
    \caption{Illustration des gleitenden Fensters anhand des Wortes Flugzeug. Die rote Linie stellt die berechneten Werte dar. Die größe des Zeitfensters zur Durchschnittsbildung konnte leider nicht rekonstruiert werden (Annahme 8, da die ersten 7 Tage nicht geplottet sind). }\label{pic.time_airplane}
\end{figure}

 

%Cleanning ##########################################
\chapter{Cleaning}

Ein Problem welches alle im vorangegangen Kapitel vorgestellten Algorithmen gemein haben ist, dass Datumsangaben, sowie Uhrzeiten und Temperaturangaben häufig hohe Relevanzwerte erhalten. Vor allem die Datumsangaben des jeweiligen Tages werden täglich als Trend erkannt. Da diese jedoch keinen Informationswert besitzen sollten sie ausgefiltert werden. Der einfachste Ansatz ist diese Werte aus der Liste der Trends zu filtern. Dies kann mittels regulärer Ausdrücke realisiert werden. Die folgenden Ausdrücke können die häufigsten Vorkommen von Datums-, Zeit- und Temperaturangaben erkennen:

\begin{itemize}
	\item{Datumsangaben der Form 01.01, 01.01.15, 01.01.2015}\\
	\begin{lstlisting}[language=sql]
'^[[:digit:]]{2}.[[:digit:]]{2}(.[[:digit:]]{2,4})?$'
	\end{lstlisting}
\end{itemize}

\begin{itemize}
	\item{Datumsangaben der Form 01. Januar}\\
	\begin{lstlisting}[language=sql]
'^[[:digit:]]{2}.[[:blank:]](Januar|Februar|März|April|Mai|Juni|Juli|August|September|Oktober|November|Dezember)$'
	\end{lstlisting}
\end{itemize}

\begin{itemize}
	\item{Zeitangaben}\\
	\begin{lstlisting}[language=sql]
'[[:digit:]]{2}:[[:digit:]]{2}'
	\end{lstlisting}
\end{itemize}

\begin{itemize}
	\item{Temperaturangaben}\\
	\begin{lstlisting}[language=sql]
'[[:digit:]]{2}[[:blank:]]?°C?'
	\end{lstlisting}
\end{itemize}

Dieser Ansatz hat einen Nachteil, da alle Datumsangaben gefiltert werden. In einigen Ausnahmefällen könnten diese jedoch tatsächlich relevante Begriffe darstellen, als Beispiel sei hier der 11. September genannt. Da es sich bei den unerwünschten Angaben jedoch meist um das aktuelle Datum bzw. einen Tag später oder früher handelt, würde es ausreichen nur diese Ausdrücke zu filtern: 

\begin{itemize}
	\item{Datumsangaben gestern, heute, vorgestern}\\
	\begin{lstlisting}[language=sql]
SET lc_time_names = 'de_DE';
CONCAT("^(",
   CONCAT_WS("|",
       DATE_FORMAT(CURDATE(),"%d.%c.%Y"),
       DATE_FORMAT(CURDATE(),"%d.%c.%y"),
       DATE_FORMAT(CURDATE(),"%d.%c"),
       DATE_FORMAT(CURDATE(),"%d. %M"),
       DATE_FORMAT(CURDATE()-1,"%d.%c.%Y"),
       DATE_FORMAT(CURDATE()-1,"%d.%c.%y"),
       DATE_FORMAT(CURDATE()-1,"%d.%c"),
       DATE_FORMAT(CURDATE()-1,"%d. %M"),
       DATE_FORMAT(CURDATE()+1,"%d.%c.%Y"),
       DATE_FORMAT(CURDATE()+1,"%d.%c.%y"),
       DATE_FORMAT(CURDATE()+1,"%d.%c")
       DATE_FORMAT(CURDATE()+1,"%d. %M"),
   ),
   ")$"
);
	\end{lstlisting}
\end{itemize}



%HAUPTTEIL IMPLEMENTIERUNG ##################################################################
%\chapter{Implementierungen in SQL und R}



% AUSWERTUNG #################################################################################
\chapter{Ein empirischer Vergleich}

Die Messung der Güte der Ergebnisse stellt eine Herausforderung dar, da es keine geeignete Referenz, beispielsweise in Form eines Goldstandards der wichtigsten Worte eines Tages gibt. Um die Güte trotzdem einschätzen zu können bieten sich zwei Herangehensweisen an. Zum einen die eigenständige manuelle Prüfung der Ergebnisse unter selbst formulierten Kriterien, zum anderen der quantitative Vergleich mittels eines geeigneten Abstandsmaßes. Letzterer Ansatz bietet aber nur die Möglichkeit eines Vergleichs der Ähnlichkeiten der Ergebnisse und hilft abzuschätzen wie sich die Ergebnisse gegeneinander verhalten. Über die Güte gibt diese Methode keine Auskunft. Allerdings lassen sich Ausreißer gut erkennen und der Prämisse, dass gleiche Ergebnisse, die aus verschiedenen Messungen stammen eine höhere Wahrscheinlichkeit besitzen gute Ergebnisse zu sein lässt sich auch die Qualität beurteilen.
\section{Qualitativer Vergleich}
Aufgrund des fehlenden Goldstandards und Expertenwissen ist ein qualitativer Vergleich nur mittels Stichproben möglich. Jedoch ist selbst bei diesem Ansatz ein Bewertung der Ergebnisse schwierig. Wir haben uns daher entschieden den Vergleich an einem Tag durchzuführen und haben den 31.10.2015 ausgewählt. 
Für diesen Tag konnten wir daher zuerst mittels der Nachrichten eine Menge an Themen als Erwartungswert zusammenstellen. Wir erwarten daher, dass Begriffe welche diesem Thema zugeordnet werden können in den jeweiligen Listen vorkommen. 
Die Themen sind: Wahl in der Türkei, Flugzeugabsturz über dem Sinai, Flchtlingskrise, 



\begin{table}
\centering
\resizebox{\textwidth}{!}{

\begin{tabular}{c|c|c|c}
\hline
Freqratio 							   & TF-IDF 					 & POISSON 						 & Z-Score \\
\hline

Zoo Leipzig               (129.6, 210) & Zoo Leipzig        (12,0, 210) & Zoo Leipzig         (69.3, 210) & Übergabe-          (7231.5,  10) \\
Flüchtlingskrise          (118.8, 124) & Gondwanaland       ( 8.2,  92) & Flüchtlinge         (67.9, 750) & Michls             ( 711.0,   8) \\
Gondwanaland              ( 94.7,  92) & Transitzonen       ( 7.6,  32) & Mohamed             (33.9, 172) & Ufo-Chef           ( 593.5,   7) \\
Transitzonen              ( 82.0,  32) & Flüchtlingskrise   ( 7.2, 124) & Gondwanaland        (32.1, 92 ) & Niedergörsdorf     ( 232.4,  16) \\
Lesbos                    ( 65.1,  62) & Graulich           ( 5.6,  42) & Zoo                 (31.6, 245) & Sachon             ( 217.3,   9) \\
lenkbare                  ( 61.9,  38) & Lesbos             ( 5.4,  62) & Mourinho            (29.2, 176) & 9268               ( 200.1,  12) \\
Hefei                     ( 51.1,  35) & Gräbersegnung      ( 4.6,  27) & Halloween           (22.9, 134) & Zoo Leipzig        ( 159.4, 210) \\
Elias                     ( 51.0, 187) & einDie             ( 4.5,  19) & Lesbos              (20.9, 62 ) & Christian Möckel   ( 120.7,   7) \\
Mohamed                   ( 42.3, 172) & Hefei              ( 4.3,  35) & Cuspert             (15.4, 72 ) & Flüchtlingskrise   ( 119.2, 124) \\
A-321                     ( 41.9,  17) & A-321              ( 4.0,  17) & Polanski            (15.2, 68 ) & Manuva             ( 118.9,   8) \\
Gräbersegnung             ( 41.8,  27) & Beilin             ( 3.8,  20) & Gasquet             (15.1, 66 ) & Roots Manuva       ( 118.9,   8) \\
Flüchtlingsjungen         ( 37.9,  21) & Hradecky           ( 3.8,  20) & Mexiko              (14.8, 207) & Flüchtlingsjunge   ( 116.4,  11) \\
Niedergörsdorf            ( 34.0,  16) & lenkbare           ( 3.8,  38) & Graulich            (14.5, 42 ) & Schlaatz           ( 110.8,  13) \\
30. Oktober               ( 33.7, 101) & Cuspert            ( 3.7,  72) & Silvio              (13.9, 93 ) & Nuthe              ( 110.6,   7) \\
Tropenerlebniswelt        ( 33.4,  27) & Scharm el Scheich  ( 3.6,  25) & Flüchtlingen        (13.7, 174) & lenkbare           ( 103.0,  38) \\
Hermanos                  ( 32.7,  32) & el Scheich         ( 3.6,  25) & Tsipras             (13.2, 74 ) & Hefei              (  93.2,  35) \\
Luftbefeuchter            ( 29.8,  21) & Okt                ( 3.5,  56) & lenkbare            (13.0, 38 ) & Flüchtlingsjungen  (  90.5,  21) \\
Spielemagazin             ( 28.7,  21) & Cryan              ( 3.5,  26) & Sinai-Halbinsel     (12.6, 56 ) & Hadassah           (  83.7,   8) \\
Graulich                  ( 27.9,  42) & Scharm             ( 3.4,  37) & Hefei               (12.4, 35 ) & Bugün              (  81.9,  11) \\
Polanski                  ( 27.5,  68) & Flüchtlingsjungen  ( 3.3,  21) & Transitzonen        (12.4, 32 ) & Gondwanaland       (  71.2,  92) \\
Klingonisch               ( 27.2,  20) & Gasquet            ( 3.2,  66) & Sinai               (12.2, 65 ) & Gräbersegnung      (  68.0,  27) \\
Balkanroute               ( 26.8,  20) & IS                 ( 2.9, 249) & Scharm              (11.4, 37 ) & Mets               (  66.8,  11) \\
Schlaatz                  ( 26.8,  13) & Niedergörsdorf     ( 2.9,  16) & Seehofer            (11.1, 177) & Lesbos             (  62.5,  62) \\
Wegscheid                 ( 26.8,  18) & Hauser-Süess       ( 2.9,  15) & Syrien              (11.0, 413) & Polański           (  62.4,  10) \\
Ägäis                     ( 26.8,  35) & studiKURIER        ( 2.8,  12) & Nadal               (10.9, 125) & Annaud             (  57.0,   8) \\
9268                      ( 26.3,  12) & Tropenerlebniswelt ( 2.8,  27) & griechischen        (10.9, 122) & Feuerwerksshow     (  54.2,  11) \\
Hauser-Süess              ( 25.9,  15) & Neubronner         ( 2.7,  17) & Gräbersegnung       (10.1, 27 ) & Hauser-Süess       (  53.7,  15) \\
Selektoren                ( 25.6,  24) & Waldhauer          ( 2.6,  18) & Passau              ( 9.5, 58 ) & Elias              (  52.5, 187) \\
Sinai                     ( 25.4,  65) & Balkanroute        ( 2.6,  20) & Blick-Abo           ( 9.3, 38 ) & Hasskommentare     (  50.8,   7) \\
Passau                    ( 25.3,  58) & Instantbird        ( 2.6,  11) & Spezialdienste      ( 9.3, 46 ) & Luftbefeuchter     (  49.9,  21) \\
Flüchtlingsjunge          ( 24.9,  11) & Kochanowski        ( 2.6,  11) & Tropenerlebniswelt  ( 8.7, 27 ) & 4Players.de        (  47.8,  42) \\
Blacks                    ( 24.5,  23) & Tsipras            ( 2.5,  74) & 31. Oktober         ( 8.6, 70 ) & Mohamed            (  46.4, 172) \\
studiKURIER               ( 24.3,  12) & 9268               ( 2.5,  12) & Lageso              ( 8.6, 35 ) & Balkan-Route       (  45.7,  11) \\
Osterspektakel            ( 24.2,  10) & Rugby-WM           ( 2.4,  19) & Scharm el Scheich   ( 8.6, 25 ) & Wegscheid          (  44.1,  18) \\
Gasquet                   ( 24.2,  66) & Parmelin           ( 2.4,  18) & el Scheich          ( 8.6, 25 ) & Hermanos           (  41.5,  32) \\
Mourinho                  ( 24.0, 176) & Halloween          ( 2.4, 134) & Leipzig             ( 8.5, 286) & Balkanroute        (  39.1,  20) \\
Blick-Abo                 ( 23.9,  38) & Overwatch          ( 2.4,  24) & Gutierrez           ( 8.5, 65 ) & News Archiv        (  38.7,  11) \\
Manuva                    ( 23.4,   8) & Lageso             ( 2.4,  35) & Cryan               ( 8.5, 26 ) & Blick-Abo          (  38.4,  38) \\
Roots Manuva              ( 23.4,   8) & Varoufakis         ( 2.3,  15) & Hermanos            ( 8.4, 32 ) & Beilin             (  38.4,  20) \\
4Players.de               ( 23.3,  42) & Dschungelnacht     ( 2.3,  16) & Klopp               ( 8.3, 153) & Bob Ross           (  37.6,  14) \\
Russisches                ( 23.1,  31) & Dupper             ( 2.3,  10) & Höttges             ( 8.2, 56 ) & 6P                 (  37.4,   7) \\
Übergabe-                 ( 22.9,  10) & Osterspektakel     ( 2.3,  10) & Verstappen          ( 8.2, 72 ) & Syrien-Gipfel      (  36.8,  11) \\
Kiwara-Savanne            ( 22.9,  15) & Schlaatz           ( 2.3,  13) & Ägäis               ( 7.8, 35 ) & EKP                (  36.2,   9) \\
Goetz                     ( 22.9,  25) & Blick-Abo          ( 2.3,  38) & Flüchtlingsjungen   ( 7.7, 21 ) & Abgastests         (  35.9,   9) \\
Sinai-Halbinsel           ( 22.8,  56) & Klingonisch        ( 2.3   20) & Liverpool           ( 7.7, 109) & Tropenerlebniswelt (  35.5,  27) \\
Bukarester                ( 22.8,  19) & Polanski           ( 2.2,  68) & 4Players.de         ( 7.6, 42 ) & Selektoren         (  35.2,  24) \\
Luckenwalde               ( 22.8,  26) & Sinai-Halbinsel    ( 2.2,  56) & Hradecky            ( 7.4, 20 ) & Kiwara-Savanne     (  34.9,  15) \\
Beilin                    ( 22.7,  20) & Wegscheid          ( 2.1,  18) & ausblenden          ( 7.4, 70 ) & Spielemagazin      (  34.7,  21) \\
Zerstörungskraft          ( 22.5,  19) & Afro-Pfingsten     ( 2.1,   9) & Beilin              ( 7.4, 20 ) & Antena             (  33.7,  11) \\
Halloween                 ( 22.3, 134) & Junhold            ( 2.1,  21) & Grip                ( 7.3, 58 ) & Schweinevogel      (  33.3,  10) \\
\hline
\end{tabular}
}
\caption{Vergleich der unterschiedlichen Algorithmen am Beispiel des 31.10.2015. Gezeigt werden die 100 relevantesten Worte und in Klammern jeweils der Score und die Tagesfrequenz}
\end{table}


\section{Quantitativer Vergleich - Average Overlap als Vergleichma\ss}


\subsection{Einf\"uhrung}
Der Vergleich zweier mit einer Rangfolge versehenen Listen ist ein bekanntes Problem. In unserem Fall handelt es sich um den spezialfall von Listen gleicher und fester L\"ange, aber einer potentiell unendlichen Zahl verschiedener W\"orter. Desweiteren sind die Listen nicht \emph{Conjoint}, was bedeutet, dass nicht nur gemeinsame W\"orter in den verschiedenen Listen auftauchen. In \cite{webber2010similarity} werden als Einleitung f\"ur ein Ma\ss, dass in der Lage ist auch unendliche Listen und Listen verschiedener L\"ange vergleichen zu k\"onnen geeignete Verfahren vorgestellt um solche Listen zu vergleichen. Das gew\"ahlte Verfahren \emph{Average Overlap} wird von den Autoren als \emph{top-k ranking} identifiziert. Also ein Ranking bis zu einer definierten Tiefe von k.\\
Der Vorteil des genutzten Verfahrens f\"ur unseren Anwendungsfall ist, dass der Rang der W\"orter einen Einfluss auf das Ma\ss haben. \"Ahnlichkeiten an der Spitze der Liste werden st\"arker gewichtet.\\
Das Verfahren ist ein Mengenbasierter Ansatz. Listen sind sich dann \"ahnlich, wenn sie die relative Anzahl gemeinsamer W\"orter hoch ist. Um nun aufsteigende Gewichtungen zu erhalten wird nun nicht nur die gesamte \"Uberlappung zweier Listen gemessen, sondern die Listen in K Listen unterteilt, wobei K die L\"ange der Listen ist und jede einzelne Liste jeweils alle Elemente bis zu dem Rang des Laufindexes k von 1 bis K enth\"alt. Also eine Liste der Form: [[ Wort 1],[Wort 1, Wort 2], ...]. Nun wird bei den einzelnen Listen gleicher L\"ange die relative \"Uberlappung gemessen. Um nun das Verlgleichsma\ss  zu erhalten wird der Durchschnitt aller errechneten Werte gemessen.  Formalisiert ergibt dies:
\begin{equation}
AO(S,T) = \frac{\sum_{k=1}^K\frac{| M(S_k) \cap M(T_k)|}{k}}{K}
\end{equation}
Wobei $S$ und $T$ zwei Listen sind, der tiefgestellte Index $k$ die Teilliste bis zum Rang $k$ angibt und $K$ die L\"ange der beiden Listen definiert. $M$ ist hierbei die Abbildung einer Liste auf die Menge der enthaltenen Elemente.\\
%Beispiel:\\
\subsection{Ergebnisse}
Hier die Ergebnisse f\"ur den 1.5.2015 mit der Listenl\"ange $K=1000$
\begin{table}[ht]
\centering
\begin{tabular}{rllr}
  \hline
 & List & List\_to\_compare & average\_overlap \\ 
  \hline
1 & tf\_idf & poisson & 0.66 \\ 
  2 & tf\_idf & z-score & 0.18 \\ 
  3 & tf\_idf & freqratio & 0.31 \\ 
  4 & tf\_idf & freqratio\_old & 0.31 \\ 
  5 & tf\_idf & poisson\_old & 0.66 \\ 
  6 & poisson & z-score & 0.15 \\ 
  7 & poisson & freqratio & 0.16 \\ 
  8 & poisson & freqratio\_old & 0.16 \\ 
  9 & poisson & poisson\_old & 1.00 \\ 
  10 & z-score & freqratio & 0.16 \\ 
  11 & z-score & freqratio\_old & 0.16 \\ 
  12 & z-score & poisson\_old & 0.15 \\ 
  13 & freqratio & freqratio\_old & 1.00 \\ 
  14 & freqratio & poisson\_old & 0.16 \\ 
  15 & freqratio\_old & poisson\_old & 0.16 \\ 
   \hline
\end{tabular}
\caption{Avarage Overlap Comparison} 
\label{AvarageOverlapComparison}
\end{table}\begin{figure}[htbp] 
  \centering
     \includegraphics[width=0.7\textwidth]{pictures/comparison.png}
  \caption{Graph of Average Overlap}
  \label{fig:comparisonGraph}
\end{figure}




% SCHLUSS #####################################################################################
\chapter{Bewertung und Zusammenfassung}\label{schluss}
Die Untersuchung der Vergleichsmaße hat das im Projekt \emph{Wörter des Tages} genutzte Poisson Vergleichsmaß als geeignet indentifiziert. Als interessante Verbesserungen wurde die Einbeziehung der Anzahl von Quellen an einem Tag, in denen ein Kandidat auftaucht. Diesen Wert in die Berechnung für ein Maß mit einzubeziehen wurde ausgeschlossen, da bei derzeitigem Datenbankschema die Berechnungzeiten gerade für höherfrequente Worte als zu lang angesehen werden. Das Z-Score Maß liefert auch gute Ergebnisse, allerdings tauchen mehr niederfrequente Worte in der Top-Wort-Liste auf.\\
Die im Abschnitt~\ref{cleaning} aufgeführten regulären Ausdrücke werden als geeignetes Mittel vorgestellt um aktuelle Datumsangaben zu filtern. Eine weitere Methode um strukturbedingt auftauchende Wörter zu filltern liefert die Einbeziehung der oben gennanten Tages-Quellenfrequenz. Tauch ein Wort in den Top-Listen auf, hat aber nur wenige Quellen, weißt das darauf hin, dass es kein Wort ist, welches das tagesaktuelle Geschehen repräsentiert. Viel mehr ist es ein Hinweis auf ein Wort, dass für dass Mass ein außergewöhnliche strukturelle Form widerspiegelt, wie beispielsweise der Name einer interviewten Person, der bei jeder Aussage im Interview wieder genannt wird und deshalb eine hohe Frequenz besitzt.\\
Ein weiteres Mittel um Informationen zu den gefundenen Wörtern aufzubereiten ist der Ansatz der Zeitreihenanalyse, der im Rahmen dieser Arbeit nicht weiter verfolgt werden konnte. Insbesondere zur Visualisierung von Frequenzverläufen einzelner Wörter scheint dies geeignet.\\
Ein weiteres in dieser Arbeit nicht weiter behandeltes Mittel um die Qualität der Ergebnisse zu verbessern ist die Überarbeitung der Liste an genutzten Quellen. Diese, so lassen interessante Wörter wie \emph{Zoo Leipzig} vermuten, haben teils einen regionalen sowie thematischen Schwerpunkt, der bei einer Anzahl von ungefähr 250 Quellen am Tag zu verzerrten Ergebnissen führen kann. Die Auswahl der Quellen scheint für folgende Arbeiten ein interessanter Untersuchungsgegenstand zu sein.



% LITERATUR ####################################################################################
%\nocite{*}%alle nicht aufgeführte Literatur auch auffuehren
\bibliographystyle{plaindin} %alphadin_martin
\bibliography{wortschatzZeitgeistLit} 

\end{document}
