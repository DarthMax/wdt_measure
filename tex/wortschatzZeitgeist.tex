\documentclass[fontsize=12pt, paper=a4, headinclude, twoside=false, parskip=half+, pagesize=auto, numbers=noenddot, open=right, toc=listof, toc=bibliography]{scrreprt}
\usepackage[left=3cm, bottom=3cm, top=3cm]{geometry} % wenn es nicht anders geht sonst typearea (unten)
\usepackage{multirow} % Tabellen-Zellen über mehrere Zeilen
\usepackage{multicol} % mehre Spalten auf eine Seite
\usepackage{tabularx} % Für Tabellen mit vorgegeben Größen
\usepackage[automark]{scrpage2} % Kopf- und Fußzeilen
\usepackage[T1]{fontenc} % Ligaturen, richtige Umlaute im PDF 
\usepackage[utf8]{inputenc}% UTF8-Kodierung für Umlaute usw
\usepackage{bibgerm} % Bibliographie Umlaute in BibTeX
\usepackage{mathtools}
\usepackage{graphicx}
\usepackage[autostyle=true,german=quotes]{csquotes} %deutsche anfuehrungszeichen
\usepackage{hyperref}
\usepackage{listings}
\usepackage{gensymb}
\usepackage{xcolor}


\lstset{
	breaklines=true, 
	postbreak=\raisebox{0ex}[0ex][0ex]{\ensuremath{\color{red}\hookrightarrow\space}},
  	literate={ö}{{\"o}}1
           {ä}{{\"a}}1
           {ü}{{\"u}}1
           {°}{{\degree}}1
}
 
 \newcommand{\ourtitlepage}{
 %++++++++++++++++++++++++++++++++++++++++++++++++++++++++++++++++++
% Titelseite
%\clearscrheadings\clearscrplain
\begin{center}
{\large Universität Leipzig}\\
{\large Fakultät für Mathematik und Informatik}\\
{\large Institut für Informatik}\\
{\large Abteilung Automatische Sprachverarbeitung}\\
\vspace{5cm}
\begin{Large}
\textbf{Vergleich von Maßen zum Finden von tagesaktuellen Wörtern}\\
\vspace{5mm}
Seminararbeit\\
\vspace{5mm}
\end{Large}
\vspace{9cm}
\begin{tabular}{ r l }
{\bf Autoren:}	& Kießling, Max\\
			& Otto, Wolfgang (2885214)\\
			& (Döring, Thomas)\\
{\bf Modul:} & Anwendungen Linguistische Informatik (10-202-2307)\\
{\bf Abgabe:} & {\today}. (Sommersemester 2015)\\
{\bf Betreuer:} & Maciej Janicki\\
{\bf Seminarleiter:}&Prof. Dr. Uwe Quasthoff\\ 
\end{tabular}\\
\end{center}
\thispagestyle{empty}
\clearpage


}


\begin{document}
\ourtitlepage 
\tableofcontents
\pagenumbering{roman} % Inhaltsverzeichnis roemisch 
\clearpage
\pagenumbering{arabic} % ab jetzt die normale arabische Nummerierung



% EINLEITUNG ###################################################################################
\chapter{Einleitung}

%##########################################
\section{Aufgabenstellung}
Das Portal \" W\"orter des Tages (wortschatz.uni-leipzig.de/wort-des-tages) stellt eine \"Ubersicht von W\"ortern, die an einem ausgew\"ahlten Tag besonders relevant erschienen dar. Die W\"orter sind in neun Kategorien eingeordnet. Nach der Beschreibung auf der Website werden die W\"orter ermittelt in dem die tagesaktuelle H\"aufigkeit eines Wortes mit seiner durchschnittlichen H\"aufigkeit \"uber l\"angere Zeit hinweg gemessen wird.\\
Die Aufgabe dieser Arbeit ist es verschiedene M\"oglichkeiten der Bestimmung von W\"ortern, die an einem gew\"ahlten Tag von besonderer Relevanz sind zu beschreiben, zu vergleichen und zu evaluieren. 
Die Datengrundlage zur erstellung der W\"orter des Tages ist ein Corpus, das durch t\"agliches crawlen von Newsseiten generiert wird. Die Quellen der Newseiten sind eine definierte Menge an f\"ur relevant erachtete Seiten mit Nachrichten wie zum Beispiel Spiegel.de.\\
Bei allen Ansätzen, die auf das Vorkommen in einem Referenzzeitraum rekurrieren besteht das Korpus aus allen gecrawlten Newsseiten des voragngegangenen Jahres (2014).
Als Zusatzaufgabe soll ein musterbasiertes Verfahren in SQL entworfen werden, das es erm\"oglicht aufgrund eines gew\"ahlten Verfahrens falsch identifizierte W\"orter zu filtern. Ein Beispiel hierf\"ur sind Datumsangaben, die als relevant erscheinen, da sie Tagesaktuell oft auftauchen, aber im Vergleichszeitraum selten.

%##########################################
\section{Status quo}

%##########################################
\section{Vergleichbare Ansätze}
Tagesaktuelle Wikiartikel\\
google trends?\\



% HAUPTTEIL THEORIE ##########################################################################
\chapter{Methoden zum Finden tagesaktueller Wörter}
Im folgenden Abschnitt werden vier Methoden vorgestellt, die f\"uer jedes Wort eines Tageskorpus eine Ma\ss zahl bestimmen, die die Relevanz des Wortes an diesem Tag ausdr\"ucken soll.

%##########################################
\section{Maße zur Trend-Detection}

%#####################
\subsection{Relatives Vorkommen (Referenz)}
\emph{Idee: } Tokens, deren relatives Auftreten am gewählten Tag im Verhältnis zum relativen Auftreten im Referenzzeitaum (2014) besonders groß ist, sind interessante Wörter.\\ 
	\emph{Formel: } 
	\begin{equation}
	sig_{freqratio}(w) = \frac{\frac{k_{day}}{n_{day}}}{\frac{k_{2014}}{n_{2014}}}
	\end{equation}
	$k_{day}$: Frequenz des Tokens an einem Tag\\
	$n_{day}$: Summe der Frequenzen aller Tokens eines Tages\\
	$k_{2014}$: Frequenz des Tokens im Referenz Zeitrahmen (2014)\\
	$n_{day}$: Summe der Frequenzen aller Tokens im Referenzzeitrahmen (2014)\\
\subsection{Poisson-Maß}

Die Formel leitet sich aus der Poissonverteilung ab und beschreibt wie Wahrscheinlich es ist, dass die gemessene Tagesfrequenz beobachtet werden kann. 
\begin{equation}
sig_{poisson}(w) = \frac{k(\log(k)-\log(n\cdot p) -1 ))}{\log(n)}
\end{equation}
k:= Anzahl der Token von w in Tagesbericht\\
n := Anzahl der Tokens in Tagesbericht\\
p := relativer Anteil eines Tokens am Jahreskorpus\\
Es ist das gleiche Maß wie in \cite[S. 338-340]{heyer06} beschrieben und hergeleitet. Hier aber nicht zum auffinden von signifikanten Kookurenzen, sondern zum auffinden von signifikanten Nennungen im Tageskorpus gegenüber einem Vergleichskorpus.\\

\subsection{Termfrequenz inverse Dokumentenfrequenz (tf-idf)}
 \begin{equation}
sig_{tf idf}(w) = \frac{k}{\max(K)} \cdot \log ( \frac{365}{|documentdays(w)|})
\end{equation}

\subsection{Termfrequenz inverse Dokumentenfrequenz inverse Quellenfrequenz (tf-idf-isf)}
\emph{Idee: } W\"orter sind dann interessant, wenn sie an einem Tag in m;glichst vielen verschiedenen Quellen gennant werden.\\
Als Quelle definieren wir die Serveradresse einer Quelle. Diese wird mittels eines regul\"aren Ausdrucks aus den zugeordneten Quellen in der MySQL-Datenbank ermittelt. Als Gesamtzahl der Quellen verwenden wir alle an einem Tag den W\"ortern zugeordnete Quellen.\\
Das entstandene Signifikanzma\ss wird wie folgt definiert:
 \begin{equation}
sig_{tf idf isf}(w) = sig_{tf idf}(w) \cdot \log ( \frac{Q_d}{q_d(w)})
\end{equation}
Analog zur inversen Dokumentenfrequenz wird also das tf-idf-Signifikanzma\ss  mit dem Logarithmus der inversen relativen Anzahl der Quellenfrequenz multipliziert. $Q_d$ ist die Anzahl aller erw\"ahnten Quellen an einem Tag $d$ und $q_d()$ bildet ein Wort auf die Anzahl der Quellen ab, in denen das Wort an Tag $d$  erw\"ahnt wird. 


%#####################
\subsection{Z-Score}
Das bei diesem, von Benattar et al. \cite{benattar2011trend} beschriebenen, Ansatz zu Grunde liegende statistische Mittel ist der Z-Score. Dieser misst die Abweichung einer Zufallsvariable vom Erwartungswert in Vielfachen der Standartabweichung. Als Zufallsvariable dient die relative Worthäufigkeit am jeweiligen Tag. Erwartungswert und Standartabweichung werden mittels des Referenzkorpus berechnet. 

\begin{itemize}
	\item{Wortfrequenz}
		\begin{align*}
			f(w)_d :&= \text{Anzahl der Vorkommen von Wort $w$ an Datum $d$}
		\end{align*}
		
	\item{relative Worthäufigkeit}\\
		Die relative Worthäufigkeit $p(w)_d$ berechnet sich durch:
		\begin{align*}
			t_d   :&= \text{Anzahl verschiedener Worte an Datum $d$} \\
			p(w)_d &= \frac{f(w)_d}{t_d}
		\end{align*}						
		
	\item{Erwartungswert}\\
		Der Erwartungswert $\bar{w}$ berechnet sich durch:
		\begin{align*}
			N      :&= \text{Anzahl der Tage im Referenzkorpus} \\
			\bar{w} &= \frac{1}{N} \sum p(w)_d
		\end{align*}
		
	\item{Standartabweichung}\\
		Die Standartabweichung $\sigma_w$ berechnet sich durch:
		\begin{align*}
			\sigma_w &= \sqrt{\frac{1}{N} \sum (p(w)_d - \bar{w}^2}
		\end{align*}
		
	\item{Z-Score}\\
		Der Z-Score $Z(w)_d$ misst die Abweichung der relativen Worthäufigkeit vom Erwartungswert in Vielfachen der Standartabweichung.
		\begin{align*}
			Z(w)_d &= \frac{p(w)_d - \bar{w}}{\sigma_w}
		\end{align*}
		
\end{itemize}

Ein Problem bei der Verwendung des Z-Scores sind Worte mit sehr kleinem Erwartungswert. Dies sind häufig Worte, welche an nur sehr wenigen Tagen und im schlimmsten Fall gar nicht im Referenzkorpus auftreten. Bei diesem Worten bedeutet bereits eine sehr geringe Wortfrequenz von ein oder zwei Vorkommen einen enormen Ausschlag im Z-Score. Um dieses sogenannte Zero-Frequency-Problem abzuschwächen schlagen Benattar et. al. vor die Worte anhand ihrer Auftrittshäufigkeit zu clustern. Den Clustern werden dabei Z-Score-Schwellwerte zugeordnet. Überschreitet der Z-Score eines Wortes den Schwellwert seines Clusters wird dieses Wort als signifikant und somit als Trend eingestuft. Cluster mit niedriger Auftrittshäufigkeit erhalten dabei höhere Schwellwerte. Je häufiger ein Wort auftritt desto niedriger wird der Schwellwert. 

\begin{itemize}
	
	\item{Auftrittshäufigkeit}\\
		Die Auftrittshäufigkeit $Po(w)$ gibt an wie vielen Tagen innerhalb des betrachteten Zeitraums das Wort mindestens einmal auftritt:
		\begin{align*}
		 	nbD(w) :&= \text{Anzahl der Tage an denen w vorkommt} \\
			c_d	   :&= \text{Anzahl der Tage innerhalb des betrachteten Zeitraums}\\
			  Po(w) &=\frac{nbD(w)}{c_d}
		\end{align*}

	\item{Schwellwerte}\\
		Die Tabelle zeigt den Z-Score-Schwellwert gecluster nach Auftrittshäufigkeit $Po(w)$:\\
		\begin{tabular}{|c|c|c|c|c|c|c|c|c|}
			\hline
			[0-5[ & [5-10[ & [10-20[ & [20-30[ & [30-50[ & [50-60[ & [60-70[ & [70-80[ & [80-100] \\
			\hline
			20 & 25 & 15 & 12 & 10 & 9 & 8 & 6 & 5 \\
			\hline
		\end{tabular}
			
\end{itemize}

%#####################
\subsection{Weitere Maße}
Einbeziehung der Anzahl von Quelle

%##########################################
\section{Zeitreihenanalysen}

%##########################################
\section{Cleaning}
Es sollen Datumsangaben und evtl. neu auftauchende strukturelle Angaben ausgefiltert werden.\\
Ansatz: Regelbasiert.\\
Gibt es Maße, die solche Angaben strukturell ausschließen?



%HAUPTTEIL IMPLEMENTIERUNG ##################################################################
\chapter{Implementierungen in SQL und R}



% AUSWERTUNG #################################################################################
\chapter{Ein empirischer Vergleich}
 
Kriterien: Anteil niederfrequenter Wörter in der Top-Liste\\
\section{Einleitung}
Die Messung der G\"ute der Ergebnisse stellt eine Herausforderung dar, da es keine geeignete Referenz, beispielswiese in Form eines Goldstandards der wichtigsten Worte eines Tages gibt. Um die G\'ute trotzdem einsch\"atzen zu k\"onnen bieten sich zwei herangehensweisen an. Zum einen die eigenst\"andige manuelle Pr\"ufung der Ergebnisse unter selbst formulierten Kriterien, zum anderen der quantitative Vergleich mittels eines geeigneten Abstandsma\ss es. Letzterer Ansatz bietet aber nur die M\"oglichkeit eines Verlgeiches der \"Ahnlichkeiten der Ergebnisse und hilft abzusch\"atzen wie sich die Ergebnisse gegeneinander verhalten. \"Uber die G\"ute gibt diese Methode keine Auskunft. Allerdings lassen sich Ausrei\ss er gut erkennen und der Pr\"amisse, dass gleiche Ergebnisse, die aus verschiedenen M\"ass ungen stammen eine h\"ohere Wahrscheinlichkeit besitzen gute Ergebnisse zu sein l\"asst sich auch die Qualit\"at beurteilen.
\section{Qualitativer Vergleich}
Um sich einen Eindruck der Ergebnisse anhand der resultierenden soriterten Wortlisten zu verschaffen wurden die Listen ausgew\"ahlter Tage verglichen. Da die Untersuchenden keine ausgewiesene Expertiese ausweist, die wichtigsten W\"orter eines t\"aglichen Nachrichtenstroms zu indentifizieren, die \"uber der eines Zeitungslesers liegt kann die Analyse nicht in die Tiefe gehen. Aber durch die Wahl der Tage l\"asst sich das \"Uberblicken der Ergebnisse vereinfachen. Deshalb w\"ahlten wir den 1.1.2015. \\
 Das funktiuoniert so noch nicht!!! Umschreiben ist nur blabla!!
\section{Quantitativer Vergleich - Average Overlap als Vergleichma\ss}
\subsection{Einf\"uhrung}
Der Vergleich zweier mit einer Rangfolge versehenen Listen ist ein bekanntes Problem. In unserem Fall handelt es sich um den spezialfall von Listen gleicher und fester L\"ange, aber einer potentiell unendlichen Zahl verschiedener W\"orter. Desweiteren sind die Listen nicht \emph{Conjoint}, was bedeutet, dass nicht nur gemeinsame W\"orter in den verschiedenen Listen auftauchen. In \cite{webber2010similarity} werden als Einleitung f\"ur ein Ma\ss, dass in der Lage ist auch unendliche Listen und Listen verschiedener L\"ange vergleichen zu k\"onnen geeignete Verfahren vorgestellt um solche Listen zu vergleichen. Das gew\"ahlte Verfahren \emph{Average Overlap} wird von den Autoren als \emph{top-k ranking} identifiziert. Also ein Ranking bis zu einer definierten Tiefe von k.\\
Der Vorteil des genutzten Verfahrens f\"ur unseren Anwendungsfall ist, dass der Rang der W\"orter einen Einfluss auf das Ma\ss haben. \"Ahnlichkeiten an der Spitze der Liste werden st\"arker gewichtet.\\
Das Verfahren ist ein Mengenbasierter Ansatz. Listen sind sich dann \"ahnlich, wenn sie die relative Anzahl gemeinsamer W\"orter hoch ist. Um nun aufsteigende Gewichtungen zu erhalten wird nun nicht nur die gesamte \"Uberlappung zweier Listen gemessen, sondern die Listen in K Listen unterteilt, wobei K die L\"ange der Listen ist und jede einzelne Liste jeweils alle Elemente bis zu dem Rang des Laufindexes k von 1 bis K enth\"alt. Also eine Liste der Form: [[ Wort 1],[Wort 1, Wort 2], ...]. Nun wird bei den einzelnen Listen gleicher L\"ange die relative \"Uberlappung gemessen. Um nun das Verlgleichsma\ss  zu erhalten wird der Durchschnitt aller errechneten Werte gemessen.  Formalisiert ergibt dies:
\begin{equation}
AO(S,T) = \frac{\sum_{k=1}^K\frac{| M(S_k) \cap M(T_k)|}{k}}{K}
\end{equation}
Wobei $S$ und $T$ zwei Listen sind, der tiefgestellte Index $k$ die Teilliste bis zum Rang $k$ angibt und $K$ die L\"ange der beiden Listen definiert. $M$ ist hierbei die Abbildung einer Liste auf die Menge der enthaltenen Elemente.\\
%Beispiel:\\
\subsection{Ergebnisse}
Hier die Ergebnisse f\"ur den 1.5.2015 mit der Listenl\"ange $K=1000$
\begin{table}[ht]
\centering
\begin{tabular}{rllr}
  \hline
 & List & List\_to\_compare & average\_overlap \\ 
  \hline
1 & tf\_idf & poisson & 0.66 \\ 
  2 & tf\_idf & z-score & 0.18 \\ 
  3 & tf\_idf & freqratio & 0.31 \\ 
  4 & tf\_idf & freqratio\_old & 0.31 \\ 
  5 & tf\_idf & poisson\_old & 0.66 \\ 
  6 & poisson & z-score & 0.15 \\ 
  7 & poisson & freqratio & 0.16 \\ 
  8 & poisson & freqratio\_old & 0.16 \\ 
  9 & poisson & poisson\_old & 1.00 \\ 
  10 & z-score & freqratio & 0.16 \\ 
  11 & z-score & freqratio\_old & 0.16 \\ 
  12 & z-score & poisson\_old & 0.15 \\ 
  13 & freqratio & freqratio\_old & 1.00 \\ 
  14 & freqratio & poisson\_old & 0.16 \\ 
  15 & freqratio\_old & poisson\_old & 0.16 \\ 
   \hline
\end{tabular}
\caption{Avarage Overlap Comparison} 
\label{AvarageOverlapComparison}
\end{table}\begin{figure}[htbp] 
  \centering
     \includegraphics[width=0.7\textwidth]{pictures/comparison.png}
  \caption{Graph of Average Overlap}
  \label{fig:comparisonGraph}
\end{figure}




% SCHLUSS #####################################################################################
\chapter{Bewertung und Zusammenfassung}



% LITERATUR ####################################################################################
\nocite{*}%alle nicht aufgeführte Literatur auch auffuehren
\bibliographystyle{plaindin} %alphadin_martin
\bibliography{wortschatzZeitgeistLit} 

\end{document}